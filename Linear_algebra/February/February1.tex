\subsection{Повторение}
\subsubsection{Линейное пространство}
Линейным пространством $V$ будем называть:\\
Модуль над полем $F$, такой что:
\[
\varphi : V \times F \to V
\]
\begin{enumerate}
    \item \( \varphi(x+y) = \varphi(x) + \varphi(y) \)
    \item \( \varphi(\alpha \cdot x) = \alpha \cdot \varphi(x) \)
\end{enumerate}
\( \forall x,y \in V, \forall \alpha \in F \)\\
Является абелевой группой по сложению, а так же все вытекающие
стандартные свойства операций сложения и умножения.

\subsubsection*{Базис векторного пространства}
Элементы линейного пространства называют векторами.\\
Базисом называют такой линейно независимый набор векторов, 
с помощью линейной комбинации которых можно выразить любой вектор
из пространства.\\
\[
    \{e_1 \dots e_i\} \in V 
\]
\[
    \forall v \in V\  \exists \{\alpha_1 \dots \alpha_i\}: \ v = \alpha_1 e_1 + \dots + \alpha_i e_i
\]
Где $e_i$ --- элементы базиса, $\alpha_i \in F$

\subsubsection*{Смена базиса}

Смена базиса — это процесс, при котором один базис векторного пространства заменяется другим.

Пусть $\{e_1, e_2, \ldots, e_n\}$ — базис векторного пространства $V$. Пусть также $\{f_1, f_2, \ldots, f_n\}$ — новый базис векторного пространства $V$, который состоит из линейно независимых векторов. Каждый вектор $f_i$ можно выразить как линейную комбинацию векторов старого базиса:

\[
f_j = \sum_{i=1}^{n} a_{ij} e_i, \quad j = 1, 2, \ldots, n
\]

где $a_{ij} \in F$ — коэффициенты, определяющие, как векторы нового базиса $f_j$ связаны с векторами старого базиса $e_i$.

Для перехода от координат, выраженных в старом базисе, к координатам в новом базисе, можно воспользоваться матрицей перехода $P$, где строки соответствуют векторам нового базиса, выраженным в старом:

\[
P = \begin{pmatrix}
a_{11} & a_{12} & \cdots & a_{1n} \\
a_{21} & a_{22} & \cdots & a_{2n} \\
\vdots & \vdots & \ddots & \vdots \\
a_{n1} & a_{n2} & \cdots & a_{nn}
\end{pmatrix}
\]

Если вектор $v$ в старых координатах записывается как $\mathbf{v}_{old} = (x_1, x_2, \ldots, x_n)^T$, то его представление в новом базисе будет:

\[
\mathbf{v}_{new} = P^{-1} \mathbf{v}_{old}
\]

где $P^{-1}$ — обратная матрица к матрице перехода.

\subsubsection*{NB}
Любой вектор в векторном пространстве может быть записан с помощью своих координат в заданном базисе. 

Пусть $V$ — векторное пространство, и $\{e_1, e_2, \ldots, e_n\}$ — его базис. Тогда любой вектор $v \in V$ может быть представлен в виде линейной комбинации векторов базиса:

\[
v = \alpha_1 e_1 + \alpha_2 e_2 + \ldots + \alpha_n e_n
\]

где $\alpha_i$ — координаты вектора $v$ в базисе $\{e_1, e_2, \ldots, e_n\}$. Обычно можно записать эти координаты вектором:

\[
\mathbf{v} = \begin{pmatrix}
\alpha_1 \\
\alpha_2 \\
\vdots \\
\alpha_n
\end{pmatrix}
\]

Таким образом, вектор $v$ может быть представлен в виде:

\[
v = \sum_{i=1}^{n} \alpha_i e_i
\]

\subsubsection*{NB}
Размерностью пространства называют количество векторов в его базисе.\\
Записывают как $\dim V$\\

\subsubsection*{Линейная оболочка}
Линейная оболочка множества векторов — это множество всех линейных комбинаций этих векторов. Если дано множество векторов $S = \{v_1, v_2, \ldots, v_k\}$ в векторном пространстве $V$, то линейная оболочка этого множества обозначается как $\text{span}(S)$ и определяется следующим образом:

\[
\text{span}(S) = \left\{v \in V: v = \alpha_1 v_1 + \alpha_2 v_2 + \ldots + \alpha_k v_k, \; \alpha_i \in F \right\}
\]

где $F$ — поле. Линейная оболочка является подпространством векторного пространства $V$. Это означает, что она содержит нулевой вектор, замкнута относительно сложения и умножения на скаляр, что сделает её подпространством $V$.

Если множество $S$ является линейно независимым, то $\text{span}(S)$ является максимальным по размерности подпространством, порождаемым векторами из $S$.

\subsubsection*{Изоморфизм векторных пространств}
Изоморфизм между двумя векторными пространствами $V$ и $W$ — это взаимно-однозначное соответствие (биекция) между их элементами, которое сохраняет операции сложения и умножения на скаляр.

Пусть $T: V \to W$ — линейное отображение. Тогда $T$ называется изоморфизмом, если выполняются следующие условия:

1. **Линейность:** Для любых векторов $u, v \in V$ и любого скаляра $c \in F$ выполняются следующие равенства:

   \[
   T(u + v) = T(u) + T(v)
   \]
   \[
   T(cu) = c T(u)
   \]

2. **Взаимно-однозначное соответствие:** Отображение $T$ является инъективным (различные векторы из $V$ отображаются в различные векторы из $W$) и сюръективным (каждому вектору из $W$ соответствует хотя бы один вектор из $V$).

Если существует изоморфизм между двумя векторными пространствами $V$ и $W$, то говорят, что эти пространства изоморфны и обозначают это как $V \cong W$. 

\subsubsection{Пространство линейных форм}
Пространство линейных форм (или пространство линейных функционалов) — это множество всех линейных отображений от векторного пространства $V$ в поле $F$.

Обозначим пространство линейных форм как $V^*$. Формально, если $V$ является векторным пространством над полем $F$, то пространство линейных форм $V^*$ определяется как:

\[
V^* = \{ f: V \to F \mid f \text{ удовлетворяет линейности: } \text{ для всех } u, v \in V, c \in F \}
\]

Каждая линейная форма $f \in V^*$ принимает вектор $v \in V$ и возвращает элемент из поля $F$. Например, линейные формы могут быть выражены с помощью скалярного произведения:

\[
f(v) = a_1 v_1 + a_2 v_2 + \ldots + a_n v_n
\]

где $a_i \in F$ — коэффициенты, а $v = (v_1, v_2, \ldots, v_n) \in V$ — вектор с координатами. 

 Структура пространства линейных форм
Пространство линейных форм $V^*$ является векторным пространством над полем $F$. Оно состоит из всех линейных комбинаций линейных функций. Если $f_1, f_2 \in V^*$, то для любых скалярных коэффициентов $\alpha_1, \alpha_2 \in F$, линейная комбинация $\alpha_1 f_1 + \alpha_2 f_2$ также является линейной формой.

 Базис и размерность
Если векторное пространство $V$ имеет базис $\{e_1, e_2, \ldots, e_n\}$, то соответствующее пространство линейных форм $V^*$ имеет размерность $n$. Базис для пространства линейных форм можно выбрать следующим образом:

\[
\{f^1, f^2, \ldots, f^n\}
\]

где $f^i$ — это линейная форма, такая что $f^i(e_j) = \delta_{ij}$, где $\delta_{ij}$ обозначает символ Кронекера.

\subsubsection*{NB}
Символ Кронекера — это специальная функция, которая принимает два целых числа и служит для определения их равенства. Символ обозначается как $\delta_{ij}$, где $i$ и $j$ — целые числа. Он определяется следующим образом:

\[
\delta_{ij} = 
\begin{cases}
1, & \text{если } i = j \\ 
0, & \text{если } i \neq j
\end{cases}
\]

\subsection{Линейный оператор}
Пусть $F$ --- поле, $V,W$ --- линейные пространства над полем.\\
Тогда $\gamma$:
\[
\gamma : V \to W
\]
\[
1. \ \gamma(v_1 + v_2) = \gamma(v_1) + \gamma(v_2)
\]
\[
2. \ \gamma(\lambda v_i) = \lambda \gamma(v_i)
\]
Называют линейным оператором.

