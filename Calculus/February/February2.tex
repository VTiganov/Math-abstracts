\subsubsection{Локальный максимум и минимум функции}
Пусть функция:
\[
    f(x): E \to \mathbb{Re}, E \subset \mathbb{Re}, x_0 \in E
\]
Определим локальный максимум(минимум) функции как:
\[
    \exists U(x_0) \subset E: \forall x \in U(x_0):
\]
\[
   1. f(x_0) \ge f(x) 
\]
\[
   2. f(x_0) \le f(x)
\]
В случаях $1.$ и $2.$ соответственно точку $x_0$ называют локальным максимумом или минимумом.

\subsubsection*{NB}
$f(x_0)$ --- локальный максимум или минимум соответственно.\\
Точка экстремума --- точка локального максимума или минимума. $(x_0)$

\subsubsection{Точка внутреннего экстремума}
Точка $x_0$ будет называться точкой внутреннего экстремума если она является экстремумом, а также:
\[
    f(x): E \to \mathbb{Re}, E \subset \mathbb{Re}, x_0 \in E
\]
$x_0$ - предельная точка $E_+$ и $E_-$\\
\( E_+ = \{x \in E | x > x_0\} \)\\
\( E_- = \{x \in E | x < x_0\} \)\\
$E_+, E_-$ --- неформально, множества, где всё либо больше, либо меньше $E$.


\subsection{Французские теоремы}

\subsubsection{Теорема Ферма}
Рассмотрим \(f(x): E \to \mathbb{Re}, E \subset \mathbb{Re}, x_0\) --- точка внутреннего экстремума,
$f(x)$ дифференцируема в $x_0$ \( \implies f'(x_0) = 0 \)
\begin{proof}
    \[ f(x_0 + h) - f(x_0) = f'(x_0)\cdot h + o(h)\]
    1. \(x_0 \text{ --- точка локального максимума } \implies f(x_0+h) - f(x_0) \le 0 \)\\
    Рассмотрим случаи \(h \to 0_+, h \to 0_-\):\\
    \(1. h \to 0_+, f'(x_0) = \lim_{h \to 0_+} \frac{f(x_0 + h) - f(x_0)}{h} \le 0 \)\\
    \(2. h \to 0_-, f'(x_0) = \lim_{h \to 0_-} \frac{f(x_0 + h) - f(x_0)}{h} \ge 0 \)\\
    \(\implies f'(x_0) = 0\)\\
    Для локального минимума доказывается аналогично, оставим на усмотрение внимательному читателю.
\end{proof}
\subsubsection*{NB} Обратное неверно.

\subsubsection{Теорема Ролля}
%%%%%%%%%%%%%%%

\subsubsection{Теорема Лагранжа и следствия}
%%%%%%%%%%%%%
\subsubsection*{Теорема}


\subsubsection*{Следствия}
1.\\
2.



\subsubsection{Теорема Коши}
%%%%%%%%%%%

\subsubsection{Теорема }

\subsection{Теорема Эйлера}
