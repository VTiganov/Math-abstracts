\subsubsection{Локальный максимум и минимум функции}
Пусть функция:
\[
    f(x): E \to \mathbb{Re}, E \subset \mathbb{Re}, x_0 \in E
\]
Определим локальный максимум(минимум) функции как:
\[
    \exists U(x_0) \subset E: \forall x \in U(x_0):
\]
\[
   1. f(x_0) \ge f(x) 
\]
\[
   2. f(x_0) \le f(x)
\]
В случаях $1.$ и $2.$ соответственно точку $x_0$ называют локальным максимумом или минимумом.

\subsubsection*{NB}
$f(x_0)$ --- локальный максимум или минимум соответственно.\\
Точка экстремума --- точка локального максимума или минимума. $(x_0)$

\subsubsection{Точка внутреннего экстремума}
Точка $x_0$ будет называться точкой внутреннего экстремума если она является экстремумом, а также:
\[
    f(x): E \to \mathbb{Re}, E \subset \mathbb{Re}, x_0 \in E
\]
$x_0$ - предельная точка $E_+$ и $E_-$\\
\( E_+ = \{x \in E | x > x_0\} \)\\
\( E_- = \{x \in E | x < x_0\} \)\\
$E_+, E_-$ --- неформально, множества, где всё либо больше, либо меньше $E$.


\subsection{Французские теоремы}

\subsubsection{Теорема Ферма}
Рассмотрим \(f(x): E \to \mathbb{Re}, E \subset \mathbb{Re}, x_0\) --- точка внутреннего экстремума,
$f(x)$ дифференцируема в $x_0$ \( \implies f'(x_0) = 0 \)
\begin{proof}
    \[ f(x_0 + h) - f(x_0) = f'(x_0)\cdot h + o(h)\]
    1. \(x_0 \text{ --- точка локального максимума } \implies f(x_0+h) - f(x_0) \le 0 \)\\
    Рассмотрим случаи \(h \to 0_+, h \to 0_-\):\\
    \(1. h \to 0_+, f'(x_0) = \lim_{h \to 0_+} \frac{f(x_0 + h) - f(x_0)}{h} \le 0 \)\\
    \(2. h \to 0_-, f'(x_0) = \lim_{h \to 0_-} \frac{f(x_0 + h) - f(x_0)}{h} \ge 0 \)\\
    \(\implies f'(x_0) = 0\)\\
    Для локального минимума доказывается аналогично, оставим на усмотрение внимательному читателю.
\end{proof}
\subsubsection*{NB} Обратное неверно.

\subsubsection{Теорема Ролля}
\( f(x): E \to \mathbb{Re}, f \in C[a,b] \) и $f$ ---дифференцируема на $(a,b)$, и \(f(a) = f(b) \implies \exists \xi \in (a,b): f'(\xi) = 0  \)

\begin{proof}
    Поскольку \(f \in C[a,b]\), то по теореме Вейерштрасса \(f\) достигает своего наибольшего и наименьшего значений на отрезке \([a, b]\).  Обозначим \(M = \max_{x \in [a,b]} f(x)\) и \(m = \min_{x \in [a,b]} f(x)\).

Рассмотрим два случая:

\begin{enumerate}
    \item \textbf{Случай 1:} \(M = m\).  В этом случае \(f(x) = const\) на \([a, b]\). Следовательно, \(f'(x) = 0\) для всех \(x \in (a, b)\).  Тогда любое \(\xi \in (a, b)\) удовлетворяет условию \(f'(\xi) = 0\).

    \item \textbf{Случай 2:} \(M > m\).  Так как \(f(a) = f(b)\), то либо \(M\) , либо \(m\) достигается во внутренней точке интервала \((a, b)\).

    \begin{itemize}
        \item Пусть \(M\) достигается в точке \(\xi \in (a, b)\), то есть \(f(\xi) = M\). Тогда \(\xi\) - точка локального максимума функции \(f\). Поскольку \(f\) дифференцируема в точке \(\xi\), то по необходимому условию экстремума \(f'(\xi) = 0\).

        \item Пусть \(m\) достигается в точке \(\xi \in (a, b)\), то есть \(f(\xi) = m\). Тогда \(\xi\) - точка локального минимума функции \(f\). Поскольку \(f\) дифференцируема в точке \(\xi\), то по необходимому условию экстремума \(f'(\xi) = 0\).
    \end{itemize}
\end{enumerate}

В любом случае, существует \(\xi \in (a, b)\) такое, что \(f'(\xi) = 0\).

Что и требовалось доказать.
\end{proof}

\subsubsection{Теорема Лагранжа и следствия}

\subsubsection*{Теорема}
Пусть \( f : [a,b] \to \mathbb{R} \) — непрерывная функция, и \( f \) дифференцируема на открытом интервале \( (a,b) \). Тогда существует такая точка \( \xi \in (a,b) \), что

\[
f'(\xi) = \frac{f(b) - f(a)}{b - a}.
\]

\begin{proof}
    Рассмотрим функцию \( g(x) \) на отрезке \( [a, b] \), определённую как:
    \[
    g(x) = f(x) - \frac{f(b) - f(a)}{b - a} (x - a).
    \]
    
    Поскольку \( f(x) \) непрерывна на \( [a, b] \) и дифференцируема на \( (a, b) \), а \( \frac{f(b) - f(a)}{b - a} (x - a) \) также непрерывна на \( [a, b] \) и дифференцируема на \( (a, b) \), то \( g(x) \) также непрерывна на \( [a, b] \) и дифференцируема на \( (a, b) \).
    
    Теперь вычислим значения \( g(a) \) и \( g(b) \):
    \[
    g(a) = f(a) - \frac{f(b) - f(a)}{b - a} (a - a) = f(a).
    \]
    \[
    g(b) = f(b) - \frac{f(b) - f(a)}{b - a} (b - a) = f(b) - (f(b) - f(a)) = f(a).
    \]
    Таким образом, \( g(a) = g(b) \).
    
    Теперь можно применить теорему Ролля к функции \( g(x) \) на отрезке \( [a, b] \). Согласно теореме Ролля, существует точка \( \xi \in (a, b) \) такая, что \( g'(\xi) = 0 \).
    
    Вычислим производную \( g'(x) \):
    \[
    g'(x) = f'(x) - \frac{f(b) - f(a)}{b - a}.
    \]
    Теперь подставим \( \xi \) в \( g'(x) \):
    \[
    g'(\xi) = f'(\xi) - \frac{f(b) - f(a)}{b - a} = 0.
    \]
    Отсюда получаем:
    \[
    f'(\xi) = \frac{f(b) - f(a)}{b - a}.
    \]
    Что и требовалось доказать.
    \end{proof}


    \subsubsection*{Следствия}

    1. Если \( f(a) = f(b) \), то \( \exists \xi \in (a,b) \) такое, что \( f'(\xi) = 0 \).
    
    \begin{proof}
    Пусть \( f(a) = f(b) \). Применим теорему Лагранжа: 
    \[
    f'(\xi) = \frac{f(b) - f(a)}{b - a} = \frac{f(a) - f(a)}{b - a} = \frac{0}{b - a} = 0.
    \]
    Таким образом, существует \( \xi \in (a,b) \) такое, что \( f'(\xi) = 0 \).
    \end{proof}
    
    2. Если функция \( f \) монотонна на \( [a, b] \), то её производная \( f' \) не меняет знака на \( (a,b) \).
    
    \begin{proof}
    Пусть \( f \) монотонна на \( [a, b] \). Если \( f \) не убывает, то \( f(a) \leq f(x) \) для всех \( x \in [a, b] \). Если бы производная \( f' \) меняла знак на \( (a,b) \), это означало бы, что существует такая точка \( \xi \in (a,b) \) такая, что \( f'(\xi) > 0 \) и \( f'(\eta) < 0 \) для какой-то \( \eta \in (a, \xi) \). Это противоречит тому, что \( f \) монотонна. Аналогично можно показать, что если \( f \) не возрастает, то её производная также не меняет знак. 
    
    Следовательно, если \( f \) монотонна, то её производная \( f' \) не меняет знака на \( (a,b) \).
    \end{proof}


    \subsubsection{Теорема Коши}

    \subsubsection*{Теорема}
    Пусть функции \( f \) и \( g \) непрерывны на отрезке \( [a, b] \) и дифференцируемы на интервале \( (a, b) \). Пусть \( g'(x) \neq 0 \) для всех \( x \in (a, b) \). Тогда существует точка \( \xi \in (a, b) \) такая, что
    \[
    \frac{f(b) - f(a)}{g(b) - g(a)} = \frac{f'(\xi)}{g'(\xi)}.
    \]
    
    \begin{proof}
    Во-первых, заметим, что \( g(b) - g(a) \neq 0 \). Если бы \( g(b) = g(a) \), то по теореме Ролля существовала бы точка \( c \in (a, b) \) такая, что \( g'(c) = 0 \), что противоречит условию \( g'(x) \neq 0 \) для всех \( x \in (a, b) \).
    
    Рассмотрим функцию \( h(x) \) на отрезке \( [a, b] \), определенную как:
    \[
    h(x) = f(x) - f(a) - \frac{f(b) - f(a)}{g(b) - g(a)} (g(x) - g(a)).
    \]
    
    Функция \( h(x) \) непрерывна на \( [a, b] \) и дифференцируема на \( (a, b) \) как комбинация непрерывных и дифференцируемых функций. Кроме того, вычислим \( h(a) \) и \( h(b) \):
    
    \[
    h(a) = f(a) - f(a) - \frac{f(b) - f(a)}{g(b) - g(a)} (g(a) - g(a)) = 0.
    \]
    
    \[
    h(b) = f(b) - f(a) - \frac{f(b) - f(a)}{g(b) - g(a)} (g(b) - g(a)) = f(b) - f(a) - (f(b) - f(a)) = 0.
    \]
    
    Таким образом, \( h(a) = h(b) = 0 \). По теореме Ролля существует точка \( \xi \in (a, b) \) такая, что \( h'(\xi) = 0 \).
    
    Теперь вычислим производную \( h'(x) \):
    \[
    h'(x) = f'(x) - \frac{f(b) - f(a)}{g(b) - g(a)} g'(x).
    \]
    
    Подставим \( \xi \) в \( h'(x) \):
    \[
    h'(\xi) = f'(\xi) - \frac{f(b) - f(a)}{g(b) - g(a)} g'(\xi) = 0.
    \]
    
    Отсюда получаем:
    \[
    f'(\xi) = \frac{f(b) - f(a)}{g(b) - g(a)} g'(\xi).
    \]
    
    Поскольку \( g'(\xi) \neq 0 \) (по условию), можно разделить обе части на \( g'(\xi) \):
    \[
    \frac{f'(\xi)}{g'(\xi)} = \frac{f(b) - f(a)}{g(b) - g(a)}.
    \]
    
    Что и требовалось доказать.
    \end{proof}


\subsection{Многочлен Тейлора}

\subsubsection{Определение}
Пусть \( f(x) \) — функция, имеющая \( n \) производных в точке \( a \). Тогда многочленом Тейлора \( n \)-ой степени для функции \( f(x) \) в точке \( a \) называется многочлен вида:

\[
T_n(x) = f(a) + \frac{f'(a)}{1!}(x-a) + \frac{f''(a)}{2!}(x-a)^2 + \cdots + \frac{f^{(n)}(a)}{n!}(x-a)^n =
\]
\[
    = \sum_{k=0}^{n} \frac{f^{(k)}(a)}{k!}(x-a)^k
\]





\subsubsection{Теорема об общем виде остатка многочлена Тейлора}


\subsubsection*{Формулировка теоремы}

Пусть функция \( f(x) \) имеет \( n+1 \) производную на отрезке \( [a, x] \). Тогда для любого \( x \) из этого отрезка существует такое число \( \xi \) между \( a \) и \( x \), что остаточный член \( R_n(x) = f(x) - T_n(x) \) можно представить в виде:

\[
R_n(x) = \frac{f^{(n+1)}(\xi)}{(n+1)!} (x-a)^{n+1}.
\]

Это называется формулой Лагранжа для остаточного члена.

\subsubsection*{Доказательство}

Для доказательства используем формулу интеграла для \( [a, x] \):

\[
f(x) = T_n(x) + R_n(x),
\]

где \( R_n(x) = f(x) - T_n(x) \).

Для начала определим функцию \( g(t) \):

\[
g(t) = f(t) - T_n(t), \quad t \in [a,x].
\]

Функция \( g(t) \) будет иметь \( n+1 \) производную. По теореме о среднем значении для дифференцируемых функций, мы можем записать:

\[
g^{(n)}(t) = f^{(n)}(t) - T_n^{(n)}(t),
\]

где \( T_n^{(n)}(t) \) — \( n \)-я производная многочлена Тейлора. 

По теореме о среднем значении, существует такое \( \xi \) в пределах \( (a, x) \) такое, что

\[
g(x) - g(a) = g'(\xi)(x-a).
\]

Применяя это n раз (для \( g', g'', \ldots, g^{(n)} \)), мы можем выразить \( R_n(x) \) как:

\[
R_n(x) = \frac{g^{(n)}(\xi)}{n!}(x-a)^{n+1} \text{ для некоторого } \xi \in (a, x),
\]

откуда

\[
R_n(x) = \frac{f^{(n+1)}(\xi)}{(n+1)!}(x-a)^{n+1}.
\]

Таким образом, теорема доказана.

\subsubsection{Остаток в форме Лагранжа}

\textbf{Формулировка:} Пусть функция \( f(x) \) имеет \( n+1 \) производную на отрезке \( [a, x] \). Тогда для любого \( x \) из этого отрезка существует такое число \( \xi \) между \( a \) и \( x \), что остаточный член \( R_n(x) = f(x) - T_n(x) \) можно представить в виде:

\[
R_n(x) = \frac{f^{(n+1)}(\xi)}{(n+1)!} (x-a)^{n+1}.
\]

\textbf{Доказательство:} 

Рассмотрим функцию:

\[
g(t) = f(t) - T_n(t),
\]

где \( T_n(t) \) — многочлен Тейлора \( n \)-ой степени в точке \( a \). Мы знаем, что \( g(a) = 0 \) и \( g^{(k)}(a) = 0 \) для \( k = 1, 2, \dots, n \). Это приводит нас к выводу о том, что можно использовать теорему о среднем значении.

Применяя теорему о среднем значении \( n \)-ого порядка, мы можем выразить остаточный член \( R_n(x) \) как:

\[
g(x) = g(a) + g'(\xi_1)(x-a) + \frac{g''(\xi_2)}{2!}(x-a)^2 + \ldots + \frac{g^{(n)}(\xi_n)}{n!}(x-a)^n,
\]

где \( \xi_i \) находятся в интервале \( (a, x) \).

Таким образом, окончательно мы запишем:

\[
R_n(x) = \frac{f^{(n+1)}(\xi)}{(n+1)!}(x-a)^{n+1},
\]

где \( \xi \) — некоторая точка на интервале \( (a, x) \). Тем самым, теорема о остатке в форме Лагранжа доказана.

\subsubsection{Остаток в форме Пеано}

\textbf{Формулировка:} Оставшийся член \( R_n(x) \) можно выразить в форме

\[
R_n(x) = o((x-a)^{n}),
\]

когда \( x \to a \). Здесь \( o((x-a)^{n}) \) означает, что остаток стремится к нулю быстрее, чем \( (x-a)^{n} \).

\textbf{Доказательство:}

Для доказательства этой формы также будем использовать выражение для остатка:

\[
R_n(x) = f(x) - T_n(x).
\]

Как мы уже показали в предыдущем разделе, остаток можно записать как:

\[
R_n(x) = \frac{f^{(n+1)}(\xi)}{(n+1)!}(x-a)^{n+1},
\]

где \( \xi \) находится между \( a \) и \( x \).

Теперь, учитывая, что \( f^{(n+1)}(x) \) остается ограниченной на отрезке \( [a, x] \):

\[
\lim_{x \to a} R_n(x) = \lim_{x \to a} \frac{f^{(n+1)}(\xi)}{(n+1)!}(x-a)^{n+1}.
\]

С учетом того, что \( (x-a)^{n+1} \) стремится к нулю быстрее, чем \( (x-a)^{n} \), можно утверждать, что 

\[
R_n(x) = o((x-a)^{n}) \quad \text{при } x \to a.
\]

Таким образом, остаток в форме Пеано также доказан.
