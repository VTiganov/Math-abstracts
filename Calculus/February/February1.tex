\vspace{0.5cm}

\subsection{Дифференцируемость функции в точке. Дифференциал.}

\Large{\textit{Определение}}

\begin{center}
    Пусть \( f: E \to \mathbb{R} \), \( E \subset \mathbb{R} \), \( x \) --- предельная точка \( E \), \( (x + h) \in E \).
\end{center}

Если

\begin{center}
    \( f(x+h) - f(x) = A(x) \cdot h + \alpha(x, h), \) \\
    при \( h \to 0, \) \\
    \( \lim_{h \to 0} \frac{\alpha(x, h)}{h} = 0, \)
\end{center}

то функция \( f \) называется \textit{дифференцируемой} в точке \( x \), а число \( A(x) \) называется \textit{производной} функции \( f \) в точке \( x \) и обозначается \( f'(x) \).

\vspace{0.5cm}

Число \( A(x) \cdot h \) называется \textit{дифференциалом} функции \( f \) в точке \( x \) и обозначается \( df(x) \).

\vspace{0.5cm}

Например, для \( f(x) = x^2 \):
\[
f(x+h) - f(x) = (x+h)^2 - x^2 = x^2 + 2xh + h^2 - x^2 = 2xh + h^2, \quad h \to 0
\]
откуда следует \( A(x) = 2x, \quad h^2 = h \cdot h = o(h) \).

\subsection{Производная}

\Large{\textit{Определение}}

\begin{center}
    Пусть \( f: E \to \mathbb{R} \), \( E \subset \mathbb{R} \), \( a \) --- предельная точка \( E \).
\end{center}

Если \( \exists \lim_{x \to a} \frac{f(x) - f(a)}{x-a}, \) то его называют производной функции в точке \( a \).

\textbf{N.B.} Пусть \( x-a = h \), то при \( x \to a \), \( h \to 0 \):
\[
\lim_{h \to 0} \frac{f(a+h) - f(a)}{h} = f'(a).
\]

\textbf{LM} (О связи производной и дифференциала)\\
\( f \text{ дифференцируема в точке } x \iff \exists \text{ конечная } f'(x). \)

\begin{proof}
    \(\implies\) (Если \( f \) дифференцируема в точке \( x \), то существует конечная \( f'(x) \)):

    По определению, если функция \( f \) дифференцируема в точке \( x \), то существует линейное приближение приращения функции:
    \[
    f(x+h) - f(x) = A(x) \cdot h + \alpha(x, h),
    \]
    где \( \alpha(x, h) \) — бесконечно малая функция, то есть \( \lim_{h \to 0} \frac{\alpha(x, h)}{h} = 0 \).

    Тогда \( A(x) \) является производной \( f \) в точке \( x \), то есть \( f'(x) = A(x) \). Следовательно, \( f'(x) \) существует и конечна.

    \(\impliedby\) (Если существует конечная \( f'(x) \), то \( f \) дифференцируема в точке \( x \)):

    Если существует конечная производная \( f'(x) \), то по определению производной:
    \[
    \lim_{h \to 0} \frac{f(x+h) - f(x)}{h} = f'(x).
    \]
    Это означает, что приращение функции можно записать в виде:
    \[
    f(x+h) - f(x) = f'(x) \cdot h + o(h).
    \]
    Следовательно, \( f \) дифференцируема в точке \( x \).
\end{proof}

\textbf{LM} Если \( f \) дифференцируема в \( x_0 \), то \( f \) непрерывна в \( x_0 \).

\begin{proof}
    По определению, если функция \( f \) дифференцируема в точке \( x_0 \), то существует предел:
    \[
    \lim_{h \to 0} \frac{f(x_0 + h) - f(x_0)}{h} = f'(x_0).
    \]
    Это означает, что приращение функции можно записать в виде:
    \[
    f(x_0 + h) - f(x_0) = f'(x_0) \cdot h + o(h).
    \]

    Теперь рассмотрим предел \( f(x_0 + h) \) при \( h \to 0 \):
    \[
    \lim_{h \to 0} f(x_0 + h) = \lim_{h \to 0} \left( f(x_0) + f'(x_0) \cdot h + o(h) \right).
    \]
    Поскольку \( f'(x_0) \cdot h \to 0 \) и \( o(h) \to 0 \) при \( h \to 0 \), получаем:
    \[
    \lim_{h \to 0} f(x_0 + h) = f(x_0).
    \]
    Следовательно, \( f \) непрерывна в точке \( x_0 \).\\
    \textit{N.B.} Не работает в обратную сторону.
\end{proof}

\textbf{N.B.} \( \tan (\alpha_\text{касательной}) = f'(x_0), \quad x_0 \text{--- точка касания.} \)

\subsection{Приближение функции, полиномы}

Приближение функции \( f(x) \) в точке \( x_0 \) в виде

\[
f(x) = \sum_{k=0}^{n} c_k (x - x_0)^k + o((x - x_0)^n), \quad x \to x_0,
\]

где 

\[
c_n = \lim_{x \to x_0} \frac{f(x) - \left( c_0 + c_1 (x - x_0) + \dots + c_{n-1} (x - x_0)^{n-1} \right)}{(x - x_0)^n}.
\]
Является приближением $f(x)$ в полиномиальном виде.

\subsection{Уравнение касательной}
Касательная к $f(x)$ в точке $x_0$ определяется уравнением вида
\begin{center}
    \(k(x) = c_0 + c_1(x-x_0)\), что\\
\end{center}
\(f(x) - k(x) = o(x-x_0), x \to x_0 \)\\

Из предыдущего пункта:\\ 
\(c_0 = \lim_{x \to x_0} f(x) = f(x_0) \),\\ 
\(c_1 = \lim_{x \to x_0} \frac{f(x) - f(x_0)}{x-x_0} = f'(x_0)\)\\
\(k(x) = f(x_0) + f'(x_0) \cdot (x-x_0)\) (Если функция дифференцируема в точке $x_0$)


\subsection{Уравнение нормали}
Уравнение нормали функции в точке $x_0$ задается так: 
\begin{center}
    \(   n(x) = f(x_0) - \frac{1}{f'(x_0)} \cdot (x-x_0)  \)
\end{center}

\subsection{Правила дифференцирования}

\textbf{Теорема}\\
Пусть \(f \text{ и } g \text{ дифференцируемы в точке } x\), тогда:\\
\begin{enumerate}
    \item \((f+g)(x) = f(x) + g(x)\)
    \item \((fg) (x) = f(x) \cdot g(x)\)
    \item \( \frac{f}{g}(x) = \frac{f(x)}{g(x)}, g \ne 0  \)
\end{enumerate}
Дифференцируемы в $x$.

\begin{enumerate}
    \item \((f+g)' = f' + g'\)
    \item \((f \cdot g)' = f' \cdot g + f \cdot g'\)
    \item \((\frac{f}{g})' = \frac{f' \cdot g - f \cdot g'}{g^2}\)
\end{enumerate}

\begin{proof}
    1. Так как \(f \text{ и } g \text{ дифференцируемы в точке } x\),\\
    \(f(x+h) - f(x) = f'(x) \cdot h + o(h), h \to 0\)\\
    \(g(x+h) - g(x) = g'(x) \cdot h + o(h), h \to 0\)\\
    Рассмотрим\\
     \((f+g)(x+h) - (f+g)(x) = \\
    f(x+h) - f(x) + g(x+h) - g(x) = f'(x) \cdot h + o(h) + g'(x) \cdot h + o(h) = h \cdot \textbf{(f'(x) + g'(x))} + o(h), h \to 0,\\
     \text{ вида } A(x) \cdot h + o(h)\)\\
    \(\implies\) дифференцируемо в $x$.\\

    3. \((\frac{f}{g})(x+h) - (\frac{f}{g})(x) = \frac{f(x+h)}{g(x+h)} - \frac{f(x)}{g(x)} = \frac{f(x+h) \cdot g(x) - f(x) \cdot g(x+h)}{g(x+h) \cdot g(x)} \)\\
    
    

    \(
    = \frac{f(x)g(x) + f'(x)g(x)h - f(x)g(x) - f(x)g'(x)h + o(h)}{g(x) \cdot g(x+h)} =
    \)\\

    \(
    = \frac{(f'(x)g(x) - f(x)g'x) \cdot h}{g(x) \cdot g(x+h)} =
    \frac{A(x) \cdot h + o(h)}{g^2(h)}, \text{ так как } h \to 0 
    \) $\implies$ дифференцируемо в $x$.\\

    ***\\
    \(
    (f(x+h) - f(x) = f'(x) \cdot h + o(h)) \implies f(x+h) = f(x) + f'(x) \cdot h + o(h)
    \)\\
    2. Доказывается аналогично частному.
\end{proof}

\subsection{Следствия}

\begin{enumerate}
    \item $(\sum_{k=0}^{n} \alpha_k \cdot f_k(x))' = \sum_{k=0}^{n} \alpha_k \cdot f'_k(x)$
    \item $(\prod_{n=1}^{k} f_n)' = f'_1 \cdot f_2 \cdot ... \cdot f_n + f_1 \cdot f'_2 \cdot ... \cdot f_n + \dots + f_1 \cdot f_2 \cdot ... \cdot f'_n$
\end{enumerate}

\subsection{Теорема. Производная композиции функций.}

\begin{center}
    \( (gf)'(x) = (g(f(x)))' = g'(f(x)) \cdot f'(x) \)
\end{center}
\(f: X \to Y, X \subset Y, f \text{ дифференцируема в } x\)\\
\(g: Y \to \mathbb{R}, Y \subset \mathbb{R}, g \text{ дифференцируема в } y = f(x) \)

\begin{proof}
    \(
    f(x+h) - f(x) = f'(x) \cdot h + o(h), h \to 0 \\
    g(y +t) - g(y) = g'(y) \cdot t + o(t), t \to 0
    \)\\

    \(
    (gf)(x+h) - gf(x) = g(f(x+h)) - g(f(x)) = \\
    =  g(f(x) + f'(x) \cdot h + o(h)) - g(f(x))
    \)\\

    ***
    \(\\
    f(x) = y \\
    \text{Возьмем за } t : f'(x) \cdot h + o(h)
    \), тогда\\
    ***\\

    \(
    g(f(x) + t) - g(f(x)) = g(y + t) - g(y) = g'(y) \cdot t + o(t) = \\
    = g'(f(x)) \cdot (f'(x) \cdot h + o(h)) + o(f'(x) \cdot h + o(h)) = \\
    = g'(x) \cdot f'(x) \cdot h + o(h), h \to 0 \\
    \text{Возьмем за } A(x) \cdot h: g'(x) \cdot f'(x) \cdot h \\
    \implies A(x) \cdot h + o(h), \text{дифференцируемо.}
    \)
\end{proof}

\subsection{Теорема. Производная обратной функции}

\(
f \text{ дифференцируема в } x_0, \exists f^{-1}(x)\\
(f^{-1})'(f(x_0)) = \frac{1}{f'(x_0)}
\)
\begin{proof}
    \(\text{ возьмем } y_0 = f(x_0)\\
    (f^{-1})' = \lim_{h \to 0} \frac{f^{-1}(y_0+h) - f^{-1}(y_0)}{h} = \\
    = \lim_{y \to y_0} \frac{f^{-1}(y) - f^{-1}(y_0)}{y - y_0} = \\
    = \lim_{x \to x_0} \frac{ f^{-1}(f(x)) - f^{-1}(f(x_0)) }{f(x) - f(x_0)} = \\
    = lim_{x \to x_0} \frac{x - x_0}{f(x) - f(x_0)} = \\
    = \frac{1}{f'(x)}
    \)
\end{proof}

\subsection*{Пример нахождения производной}
\[
(x^n)' = nx^{n-1}
\]
Рассмотрим определение производной:
\[
\lim_{h \to 0} \frac{(x+h)^n - x^n}{h}.
\]
Вынесем $x^n$:
\[
(x+h)^n = x^n \left(1 + \frac{h}{x}\right)^n.
\]
Тогда:
\[
\lim_{h \to 0} \frac{x^n \left( (1 + \frac{h}{x})^n - 1 \right)}{h}.
\]
Используем приближенное разложение \((1 + u)^n \approx 1 + nu + o(u)\) при малых \( u \):
\[
(1 + \frac{h}{x})^n - 1 \approx n \frac{h}{x} + o(h).
\]
Подставляем:
\[
\lim_{h \to 0} \frac{x^n \left(n \frac{h}{x} + o(h)\right)}{h}.
\]
Раскрываем множители:
\[
\lim_{h \to 0} \frac{n x^{n-1} h + x^n o(h)}{h}.
\]
Разделяем дробь:
\[
\lim_{h \to 0} \left( n x^{n-1} + x^n \frac{o(h)}{h} \right).
\]
Так как \( \frac{o(h)}{h} \to 0 \) при \( h \to 0 \), остается:
\[
n x^{n-1}.
\]
Следовательно, производная:
\[
(x^n)' = n x^{n-1}.
\]

\subsection{Параметрические функции}

\[
\left\{
\begin{array}{l}
    x(t) = \cos(t) \\
    y(t) = \sin(t)
\end{array}
\right.
\]
\begin{center}
    \(
    y'_x = \frac{y'_t}{x'_t} = \frac{\cos(t)}{- \sin(t)} = - \ctg(t)
    \)
\end{center}
