\subsection{Аппроксимация функции и смежные вопросы}
Пусть имеем $f(x)$ --- функцию. Заменять или приближать ее будем 
в случае, если исходная $f(x)$ неудобна для нужных операций (дифференцирование,
интегрирование и другие)\\
Самый распространенный способ приближения - с помощью полиномов,
так как они достаточно простые и имеют понятное поведение.
\\

$f(x)$ задана значениями: 
\[
x_1 \dots x_i
\]
для них известны
\[
f(x_1) \dots f(x_i)
\]
будем приближать обобщенным многочленом вида:
\[
Q_m(x) = a_0\varphi_0(x) + a_1\varphi_1(x) + ... + a_m\varphi_m(x) = \sum_{k=0}^{m} a_k\varphi_k(x)
\]
При этом $\{\varphi_k\}$ --- заданный набор ЛНЗ функций, а $a_k$ подлежат определению.
\begin{equation}
    Q_m(x_i) = f(x_i), i=0,1,\dots,m.    
\end{equation}

Тогда $Q_m(x)$ называется интерполяционным многочленом, а $x_k$ --- узлами интерполирования.
\\
\subsubsection{Интерполяционный полином Лагранжа.
Остаточный член полинома Лагранжа}
Будем рассматривать полиномиальную аппроксимацию, а именно с помощью полинома Лагранжа.
\\

Однако, не будем брать в качестве
\( \varphi_k(x)=x^k \)
так как это сильно затруднит решение системы $(1)$.\\

Выберем полиномы $\omega_0(x), \omega_1(х), ... , \omega_m(x)$ такие, что они обращаются в нуль во всех узлах, кроме одного, когда
индекс полинома совпадает с индексом узла:
\[
\omega(x) = (x-x_0)(x-x_1)\dots(x-x_m)
\]
\[
\omega_k(x) = \frac{\omega(x)}{(x-x_k)} =(x-x_0)(x-x_1)\dots(x-x_{k-1})(x-x_{k+1})\dots(x-x_m) 
\]
И запишем следующий полином:
\[
Q_m(x) = \sum_{k=0}^{m} \frac{\omega_k(x)}{\omega_k(x_k)} f(x_k)
\]
Полученный полином называют \textit{интерполяционным полиномом Лагранжа.}