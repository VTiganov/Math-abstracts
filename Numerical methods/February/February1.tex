
\subsection{Формула интерполяции Лагранжа}
Пусть заданы \(n+1\) различных точек данных
\[
(x_0, y_0), \quad (x_1, y_1), \quad \dots, \quad (x_n, y_n).
\]
Интерполяционный полином Лагранжа \(P(x)\) определяется как
\[
P(x) = \sum_{i=0}^{n} y_i L_i(x),
\]
где \emph{базисные полиномы Лагранжа} \(L_i(x)\) определяются следующим образом:
\[
L_i(x) = \prod_{\substack{j=0 \\ j \neq i}}^{n} \frac{x - x_j}{x_i - x_j}.
\]
Каждый базисный полином \(L_i(x)\) удовлетворяет свойству
\[
L_i(x_j) = \begin{cases}
1 & \text{если } i = j, \\
0 & \text{если } i \neq j.
\end{cases}
\]
Это свойство гарантирует, что при \(x = x_i\) единственный ненулевой член в сумме --- \(y_i L_i(x_i)=y_i\), то есть \(P(x_i)=y_i\).

\subsubsection{Вывод и объяснение}
Основная идея интерполяции Лагранжа заключается в построении базисных полиномов \(L_i(x)\), которые действуют как <<переключатели>>. Для заданного \(x_i\) имеем:
\[
L_i(x_j) =
\begin{cases}
1, & \text{если } j=i, \\
0, & \text{если } j\neq i.
\end{cases}
\]
Таким образом, для любого \(x_k\)
\[
P(x_k) = \sum_{i=0}^{n} y_i L_i(x_k) = y_k.
\]
Это подтверждает, что полином \(P(x)\) точно интерполирует заданные точки данных.

\subsubsection{Пример}
Рассмотрим следующий набор точек:
\[
(1,2), \quad (3,4), \quad (5,6).
\]
Для этих трёх точек интерполяционный полином будет иметь степень не более 2. Базисные полиномы Лагранжа имеют вид:

\[
L_0(x) = \frac{(x-3)(x-5)}{(1-3)(1-5)} = \frac{(x-3)(x-5)}{8},
\]
\[
L_1(x) = \frac{(x-1)(x-5)}{(3-1)(3-5)} = \frac{(x-1)(x-5)}{-4},
\]
\[
L_2(x) = \frac{(x-1)(x-3)}{(5-1)(5-3)} = \frac{(x-1)(x-3)}{8}.
\]

Таким образом, интерполяционный полином имеет вид:
\[
P(x) = 2\cdot L_0(x) + 4\cdot L_1(x) + 6\cdot L_2(x).
\]
Раскрытие скобок и упрощение этого выражения даст явную форму полинома, проходящего через заданные точки.

\subsubsection{Ошибки вычисления и функция \(\psi\)}
В полиномиальной интерполяции ошибка (или остаток) при приближении функции \(f(x)\) её интерполяционным полиномом \(P(x)\) может быть выражена следующим образом:
\[
f(x) - P(x) = \frac{f^{(n+1)}(\xi)}{(n+1)!}\psi(x),
\]
где \(\xi\) --- некоторая точка на отрезке, содержащем \(x\) и узлы интерполяции, а
\[
\psi(x) = \prod_{i=0}^{n} (x - x_i)
\]
является \emph{полиномом узлов} (иногда его также обозначают \(\omega(x)\)). Функция \(\psi(x)\) играет ключевую роль в оценке ошибок:
\begin{itemize}
    \item \textbf{Малое значение \(\psi(x)\):} Когда \(x\) близко к одному из узлов \(x_i\), один или несколько множителей \((x-x_i)\) малы, что приводит к малому значению \(\psi(x)\) и, соответственно, к низкой ошибке интерполяции.
    \item \textbf{Большое значение \(\psi(x)\):} Когда \(x\) далеко от узлов интерполяции, \(\psi(x)\) может становиться большим, усиливая ошибку.
\end{itemize}

\subsubsection*{Численные ошибки при вычислении}
При реализации интерполяционного полинома Лагранжа на практике необходимо учитывать возможные численные проблемы:
\begin{itemize}
    \item \textbf{Округлительные ошибки:} Вычисление произведений и делений --- особенно при построении базисных полиномов \(L_i(x)\) или функции \(\psi(x)\) --- может приводить к накоплению ошибок округления, особенно для полиномов высокой степени.
    \item \textbf{Явление Рунге:} При использовании равномерно распределённых узлов интерполяционный полином может демонстрировать сильные колебания около концов отрезка, что приводит к значительным ошибкам вычисления.
\end{itemize}


